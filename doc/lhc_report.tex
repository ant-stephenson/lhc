\documentclass[a4paper,11pt, notitlepage]{article}
%\usepackage[utf8]{inputenc}

\usepackage[margin=1.4in]{geometry}

\usepackage{textcomp}
\usepackage{inputenc}

\usepackage{color}

\usepackage{hyperref}
\hypersetup{colorlinks=true, linkcolor=blue}

\usepackage{cite}

\usepackage[ruled,vlined]{algorithm2e}

\usepackage{graphicx}
\usepackage{caption}
% \usepackage{subcaption}
\DeclareGraphicsExtensions{.png, .pdf, .jpeg, .jpg}
\usepackage{adjustbox}
\usepackage{subfig}

\usepackage{amsmath}
\usepackage{bm}
\DeclareMathOperator*{\argmax}{argmax}
\DeclareMathOperator*{\argmin}{argmin}
\DeclareMathOperator{\Tr}{Tr}
\DeclareMathOperator{\Cov}{Cov}
\DeclareMathOperator{\Var}{Var}
\DeclareMathOperator{\E}{E}
\DeclareMathOperator\supp{supp}
\DeclareMathOperator{\sign}{sign}
\DeclareMathOperator{\const}{const.}
\DeclareMathOperator{\diag}{diag}
\DeclareMathOperator{\st}{subject\:to}

\usepackage{amssymb}
\usepackage{wasysym}
\usepackage{mathtools}
\usepackage{gensymb}
\usepackage{tensor}
%\usepackage{langle|}
\usepackage{bbold}

\usepackage{wrapfig}

%eqn environments
\newcommand{\be}{\begin{equation}}
\newcommand{\ee}{\end{equation}}
\newcommand{\eq}[1]{\begin{align*}#1\end{align*}}
\newcommand{\numeq}[1]{\begin{align}#1\end{align}}
\newcommand{\bp}{\begin{pmatrix}}
\newcommand{\ep}{\end{pmatrix}}

%vectors
\newcommand{\mc}[1]{\mathcal{#1}}
\newcommand{\wv}{\mathbf{w}}
\newcommand{\xv}{\mathbf{x}}
\newcommand{\Xv}{\mathbf{X}}
\newcommand{\yv}{\mathbf{y}}
\newcommand{\zv}{\mathbf{z}}
\newcommand{\Iv}{\mathbf{I}}
\newcommand{\Kv}{\mathbf{K}}
\newcommand{\kv}{\mathbf{k}}
\newcommand{\phiv}{\bm{\phi}}
\newcommand{\Phiv}{\mathbf{\Phi}}
\newcommand{\betav}{\bm{\beta}}
\newcommand{\muv}{\bm{\mu}}
\newcommand{\Sigmav}{\mathbf{\Sigma}}
\newcommand{\norm}[1]{\left\lVert#1\right\rVert}

%other maths
\newcommand{\pd}{\partial}
\newcommand{\inprod}[2]{\langle#1,#2\rangle}
\newcommand{\half}{\frac{1}{2}}
\newcommand{\e}{\mathrm{e}}
\newcommand{\eps}{\epsilon}

%statistics
\newcommand{\mvnprefactor}{\frac{1}{(2\pi)^{d/2}|\Sigmav|^{1/2}}}

%probability
\newcommand\given[1][]{\:#1\vert\:}


\newcommand{\convexpath}[2]{
[   
    create hullnodes/.code={
        \global\edef\namelist{#1}
        \foreach [count=\counter] \nodename in \namelist {
            \global\edef\numberofnodes{\counter}
            \node at (\nodename) [draw=none,name=hullnode\counter] {};
        }
        \node at (hullnode\numberofnodes) [name=hullnode0,draw=none] {};
        \pgfmathtruncatemacro\lastnumber{\numberofnodes+1}
        \node at (hullnode1) [name=hullnode\lastnumber,draw=none] {};
    },
    create hullnodes
]
($(hullnode1)!#2!-90:(hullnode0)$)
\foreach [
    evaluate=\currentnode as \previousnode using \currentnode-1,
    evaluate=\currentnode as \nextnode using \currentnode+1
    ] \currentnode in {1,...,\numberofnodes} {
  let
    \p1 = ($(hullnode\currentnode)!#2!-90:(hullnode\previousnode)$),
    \p2 = ($(hullnode\currentnode)!#2!90:(hullnode\nextnode)$),
    \p3 = ($(\p1) - (hullnode\currentnode)$),
    \n1 = {atan2(\y3,\x3)},
    \p4 = ($(\p2) - (hullnode\currentnode)$),
    \n2 = {atan2(\y4,\x4)},
    \n{delta} = {-Mod(\n1-\n2,360)}
  in 
    {-- (\p1) arc[start angle=\n1, delta angle=\n{delta}, radius=#2] -- (\p2)}
}
-- cycle
}

\usepackage[toc,page]{appendix}
\usepackage{fancyhdr}
\usepackage{enumitem}
\usepackage{pifont}
% \setlist{nolistsep,leftmargin=*}

\pagestyle{fancy}
\rhead{}
\lhead{}

\author{Mansell, Georgie \and Stephenson, Ant}
\date{\today}
\title{Something something Higgs boson}

\begin{document}
\maketitle
\begin{abstract}
    ...
\end{abstract}

\section{Introduction}
After the theoretical discovery of the Higgs mechanism and the associated particle, the Higgs boson, experimental confirmation of the prediction became a much sought-after prize in High Energy Physics. Although the theory was more or less accepted by the theoretical community since its inception in the 1960s, experimental evidence was required for verification. 

Once CERN's Large Hadron Collider (LHC) emerged on the scene, with the necessary power to probe higher energy regimes of collisions, vast amounts of data started to be collected. In order to verify the theoretical predictions of particle physics, this data needed to be anaylsed carefully to try and distinguish signal of as-yet unknown particle decays vs effectively ``proven'' existing decays. The accepted threshold for a discovery in particle physics is the ``gold-standard'' 5-sigma rule. As a result, any announcement of a new experimental discovery (such as the Higgs boson) was required to meet this threshold. 

For the Higgs boson in particular, various decay mechanisms were proposed that could be used to demonstrate its existence by comparing the rate of the by-products of those decays vs background mechanisms (that do not involve the Higgs). In 2013, when the Higgs was experimentally discovered by CERN, the evidence was provided by \emph{bosonic} decays from the Higgs to the following pairs, $\gamma\gamma$, $WW$ and $ZZ$. 

The goal here is to instead examine the coupling of the Higgs to fermions, to verify that their mass can likewise be explained by the Higgs mechanism. Specifically, here the aim is to analyse the decay $H\rightarrow\tau\tau$; i.e. the decay of the Higgs to a pair of tau-flavour leptons. (The other candidates, electrons and muons fall outside the energy range of the LHC due to their lighter mass).

\section{Theoretical Background}
\subsection{Physics}
Not sure how much we actually need to include.

\subsection{Statistics}
Likewise.

\section{Method}

\subsection{Missing Data}
Early examinations of the dataset revealed the presence of a significant quantity of missing data. Elements of the data matrix recorded as missing were labelled by a value of -999. Before embarking on any model building or even feature engineering we analysed the missing values to try and ascertain whether there was any pattern to their locations. From the definitions of covariates in the dataset it was possible to infer potential causes of missing data. In particular, the estimated mass of the Higgs ($DER\_mass\_MMC$) ``(may be undefined if the topology of the event is too far from the expected topology)", indicates that this feature may be expected to be labelled as missing. Similarly, a set of the features mention a dependence on the number of jets measured in the interaction:

\begin{description}
    \item[$DER\_deltaeta\_jet\_jet$] undefined if $PRI\_jet\_num \leq 1$
    \item[$DER\_mass\_jet\_jet$] undefined if $PRI\_jet\_num \leq 1$
    \item[$DER\_prodeta\_jet\_jet$] undefined if $PRI\_jet\_num \leq 1$
    \item[$PRI\_jet\_subleading\_pt$] undefined if $PRI\_jet\_num \leq 1$
    \item[$PRI\_jet\_subleading\_phi$] undefined if $PRI\_jet\_num \leq 1$
    \item[$PRI\_jet\_subleading\_eta$] undefined if $PRI\_jet\_num \leq 1$
    \item[$DER\_lep\_eta\_entrality$] undefined if $PRI\_jet\_num \leq 1$
    \item[$PRI\_jet\_leading\_pt$] undefined if $PRI\_jet\_num = 0$
    \item[$PRI\_jet\_leading\_eta$] undefined if $PRI\_jet\_num = 0$
    \item[$PRI\_jet\_leading\_phi$] undefined if $PRI\_jet\_num = 0$   
\end{description}

So we can see that the missing data is explained by the combination of an unexpected topology and the number of jets observed. This implies that the physical data generating process is different in each of these regimes and that we ought to treat them differently in accordance. As such we chose to attempt to build a separate model for each regime. 

\subsection{Performance Metrics}
\subsubsection{AMS}

\subsubsection{ROC and AUC}

\subsection{Logistic Regression}

\subsection{SVM}

\subsection{Feature Engineering}
In order to improve the performance of our model, we attempted to carry out some feature engineering to extract as much information as possible from the dataset.

\subsubsection{Redundancy}
By considering the basic physics of the beam it is possible to see that there is some redundancy amongst the (primitive) features in the dataset. We can exploit this fact and reduce our initial feature space slightly, by transforming the redundant features into a set of new derived features that contain the same information. 

\subsubsection{Higher-Order Effects}
To attempt to capture generic higher order interaction-type effects, we implemented a set of RBF centroid features. The implementation works as follows 
\begin{algorithm}[H]
    1. Generate heuristic estimate of RBF hyperparameter $s$.\\
    2. Select $n_c$ points from $\Xv_{train}$, $\Xv_c = \{\xv^{train}_i, \dots, \xv^{train}_{n_c}\}$.\\
    3. \For{i=1:$n_c$}{
        Calculate $\kv_{RBF}(\xv^{train}_i, \Xv_{train}; s)$\\
        $\Xv_{train} \leftarrow \Xv_{train} \oplus \kv_{RBF}(\xv^{train}_i, \Xv_{train}; s)$
    } 
    4. Fit model.\\
    5. \For{i=1:$n_c$}{
        Calculate $\kv_{RBF}(\xv^{train}_i, \Xv_{test}; s)$\\
        $\Xv_{test} \leftarrow \Xv_{test} \oplus \kv_{RBF}(\xv^{train}_i, \Xv_{test}; s)$
    }
    \caption{Augment covariate matrix with RBF centroid features}
\end{algorithm}

\subsubsection{Transformations}

\subsubsection{Interactions}

\section{Results}
\subsection{Feature Engineering}

\subsection{Model Selection}

\subsection{Predictions}
To actually calculate AMS scores, we need to convert out probabilistic outputs into binary labels which means we need to pick a decision rule, i.e. a threshold over which we assign a label 1 (or $s$) vs 0 (or $b$).

\section{Conclusion}

\pagebreak 


\begin{appendices}
\section{AMS}
\label{appendix:ams}
Is the AMS metric convex?
\eq{
    \rm{AMS} &= \sqrt{2\left((s+b)\log(1+\frac{s}{b})-s\right)}
}

In the documentation, the implication is that in general, we expect $b \gg s$ giving the approximate AMS as $\rm{AMS} \sim \frac{s}{\sqrt{b}}(1+\mc{O}(\frac{s}{b}))$.

If we directly check the derivatives of the approximate AMS in the large $b$ regime:
\eq{
    \pd_s^2\frac{s}{\sqrt{b}} &= 0\\
    \pd_b^2\frac{s}{\sqrt{b}} &= \frac{3}{4}\frac{s}{b^{5/2}} >0\\
    \pd_{bs}^2 &= -\frac{1}{2b^{3/2}} < 0
}
which implies non-convexity (and non-concavity).

To check more carefully though, calculate the terms of the Hessian of the original and then apply the approximation:
\eq{
    \pd_s\rm{AMS} &= \rm{AMS}^{-1}\log\left(1+\frac{s}{b}\right) \\
    \pd_b\rm{AMS} &= \rm{AMS}^{-1}\left(\log(1+\frac{s}{b}) -\frac{s}{b}\right) \\
    \pd_s^2\rm{AMS} &= \rm{AMS}^{-1}\left(\frac{1}{b+s} - \rm{AMS}^{-2}\log(1+\frac{s}{b})^2\right) \\
    \pd_b^2\rm{AMS} &= \rm{AMS}^{-1}\left(\frac{s^2}{b^2(s+b)} -\rm{AMS}^{-2}(\log(1+\frac{s}{b}) -\frac{s}{b})^2\right) \\
    \pd_{sb}^2\rm{AMS} &= -\rm{AMS}^{-1}\left(\frac{s}{b(s+b)} - \rm{AMS}^{-2}\log(1+\frac{s}{b})(\log(1+\frac{s}{b}) -\frac{s}{b})\right)
}
and then verify whether any of them are ever $<0$.

Using the AMS approximation, and $\log\left(1+\frac{s}{b}\right) \sim \frac{s}{b} - \half\left(\frac{s}{b}\right)^2$ we get the following
\eq{
    \pd_s^2\rm{AMS} &\sim \frac{\sqrt{b}}{s}\left[\frac{1}{(b+s)} - \frac{b}{s^2}(\frac{s^2}{b^2} - \frac{s^3}{b^3})\right] \\
    &= \frac{\sqrt{b}}{s(s+b)} - \frac{1}{s\sqrt{b}} + \frac{1}{b^{3/2}} \\
    &\gtrsim \eps
}
Since in this regime of large $b$ this term is negative, the Hessian cannot be positive definite and hence the metric is non-convex. If the Hessian is \emph{concave} though, we can simply optimize $-\rm{AMS}$, so we need to check the other terms.

\eq{
    \pd_b^2\rm{AMS} &\sim \frac{\sqrt{b}}{s}\left[\frac{s^2}{b^2(s+b)} - \frac{b}{s^2}(\frac{1}{4}\frac{s^4}{b^4})\right] \\
    &= \frac{s}{b^{3/2}(s+b)} - \frac{1}{4}\frac{s}{b^{5/2}} \\
    &\gtrsim \eps
}

\eq{
    \pd_{bs}^2\rm{AMS} &\sim -\frac{\sqrt{b}}{s}\left[\frac{s}{b(s+b)} + \frac{b}{s^2}(\frac{s}{b} -\half\frac{s}{b^2})(-\half\frac{s^2}{b^2})\right] \\
    &= \frac{1}{2b^{3/2}}(1-\half s) - \frac{1}{\sqrt{b}(s+b)} \\
    &< 0
}

where we take $\eps$ to be some suitably small positive number.

If we decide to take some tolerance, $\eps \sim 10^{-6}$ at which to ignore higher order terms in $\frac{s}{b}$ we can check specifically that the claims on the Hessian above hold in that regime.

So, let's assume we can ignore terms of $O(\frac{s}{b})^3$ then for $\eps\sim 10^{-6}$ a ratio of $\frac{s}{b}\sim 10^{-2}$ would suffice; so to test this, let us choose $s=1$ and $b=100$ for simplicity.
With this, we find that $\pd_s^2\rm{AMS} \sim +10^{-6}$, $\pd_b^2\rm{AMS} \sim +10^{-6}$ and $\pd_{bs}^2\rm{AMS} \sim -10^{-4}$ which satisfies the claims above.
Finally, to verify that this is a reasonable assertion, if we check the ratio of $s:b$ in the data (using the weight vector) we find a ratio of approximately $10^{-3}$, which means our assumption was conservative. 


\section{Modelling}

\subsection{Newton's Method for Logistic Regression}
\label{appendix:newton_lr}
From the conditional distribution under the logistic regression model for a single point $\xv \in R^d$, $y \in \{+1,-1\}$ and coefficients $\betav \in R^d$.
\eq{
    p(y\given \xv, \betav) &= \sigma(f(\xv;\betav)\cdot y) \\
    &= (1 + \exp(y\inprod{\xv}{\betav}))^{-1}
}

We want to minimise the negative loglikelihood:
\eq{
    l(\xv, y, \betav) &= \log(1 + \e^{y\inprod{\xv}{\xv}}) \\
    \pd_{\beta_a} l &= \frac{yx_a\e^{y\inprod{\xv}{\betav}}}{1 + \e^{y\inprod{\xv}{\betav}}} \\
    \pd^2_{\beta_a \beta_b}l &= \frac{y^2x_ax_b\e^{y\inprod{\xv}{\betav}}}{(1+\e^{y\inprod{\xv}{\betav}})^2}\left( 1 + \e^{y\inprod{\xv}{\betav}} - \e^{y\inprod{\xv}{\betav}}\right) \\
    &= \frac{x_ax_b\e^{y\inprod{\xv}{\betav}}}{(1+\e^{y\inprod{\xv}{\betav}})^2}\\
    &= x_ax_bw(x)
}
where in the last line we define a weight function $w(x) = \frac{\e^{-x}}{(1+\e^{-x})^2}$ (and the elementwise vector version, $\wv(\xv) = \frac{\e^{-\xv}}{(1+\e^{-\xv})^2}$).

To implement Newton's method we want to try and simplify the expressions. First define the logistic and logit functions as
\eq{
    \gamma(x) &= (1+\e^{-x})^{-1} \\
    \gamma^{-1}(p) &= \log\left(\frac{p}{1-p}\right)
}
and then write out the derivatives in terms of the logistic function:
\eq{
    \pd_{\beta_a} l(y=+1) &= x_a\gamma(\inprod{\xv}{\betav}) \\
    \pd_{\beta_a} l(y=-1) &= -x_a(1-\gamma(\inprod{\xv}{\betav})
}
If we then map $y \in \{1,-1\}$ to $y' \in \{0,1\}$ we can condense these into a single expression
\eq{
    \pd_{\beta_a} l(y) &= -x_a(y - \gamma(\inprod{\xv}{\betav}))
}
and rewrite the loss as $l(y') = \log(1+\e^{(-1)^{y'}\inprod{\xv}{\betav}})$.

Now for our entire dataset $D$, assumed to be IID, we have a total loss, gradient and Hessian
\eq{
    L &= \sum_{i \in D}l_i = \sum_{i \in D}\log(1+\e^{y_i\inprod{\xv_i}{\betav}})\\
    \pd_{\betav}L &= -\sum_{i \in D}\xv_i(y_i - \gamma(\inprod{\xv_i}{\betav})) \\
    &= -\Xv^T(\yv-\gamma(\Xv\betav)) \\
    \bm{H} &= \sum_{i \in D}\xv_i\xv_i^Tw(\xv_i)\\
    &= \Xv^T\bm{W}\Xv
}
with $\Xv = \bp \xv_1^T \\ \vdots \\ \xv_n^T \ep \in R^{n\times d}$, $\bm{W} = \diag(\wv(\xv)) \in R^{n\times n}$.

\subsubsection{Regularisation}
We can add constraints to our minimisation problem, to help constrain the coefficients and effectively implement $L_2$ regularisation by formulating the following convex optimisation problem.
\eq{
    \min. &\quad L \\
    \st &\quad \norm{\betav}^2\leq c
}
From this we can construct the Lagrangian $\mc{L} = L + \lambda(\norm{\betav}^2 - c)$ and hence augment the gradient and Hessian we calculated above as
\eq{
    \pd_{\betav}\mc{L} &= \pd_{\betav}L + 2\lambda\betav \\
    \bm{\tilde{H}} &= \bm{H} + 2\lambda\Iv
}

Since we have a free parameter $c$ that determines the optimal value $\lambda^\star(c)$ of our dual variable, we can equivalently ignore $c$ and treat $\lambda$ as a parameter to be tuned.

\subsubsection{Newton's method}

\begin{algorithm}[H]
    1. Initialise parameters; $\betav_0$, $\lambda_0$, $\eps$, $L_0 = L(\betav_0)$, $m$\\
    2. \For{i=1:m}{
        Compute $\pd_{\betav}L$ and $\bm{H}$\\
        Calculate Newton step $\Delta\betav = -\bm{H}^{-1}\pd_{\betav}L$\\
        Calculate stepsize $\alpha$ with backtracking linesearch (\ref{algo:line}) \\
        Update $\betav \leftarrow \betav + \alpha\Delta\betav$ \\
        Compute new loss $L$ \\
        Test stopping criterion: $|L-L_0| < \eps$
    } 
    3. Return $\hat{\betav}$.\\
    \caption{Newton's method}
\end{algorithm}
\label{algo:newton}

\begin{algorithm}[H]
    1. Initialise parameters; $\gamma, \tau$ \\
    2. Set largest stepsize $s_{max}\leftarrow 1$ \\
    3. Set $s \leftarrow s_{max}$ \\
    4. \While{$f(x+\Delta x) > f(x) + \gamma s \nabla f^T\Delta x$} {
        Update $s\leftarrow \tau s$ \\
    }
    5. Return $s$
    \caption{Backtracking linesearch}
\end{algorithm}
\label{algo:line}

\end{appendices}
\bibliography{lhc.bib}{}
\bibliographystyle{plain}

\end{document}