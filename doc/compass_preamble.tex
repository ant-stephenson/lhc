\documentclass[a4paper,11pt, notitlepage]{article}
%\usepackage[utf8]{inputenc}

\usepackage[margin=1.4in]{geometry}

\usepackage{textcomp}
\usepackage{inputenc}

\usepackage{color}

\usepackage{hyperref}
\hypersetup{colorlinks=true, linkcolor=blue}

\usepackage{cite}

\usepackage[ruled,vlined]{algorithm2e}

\usepackage{graphicx}
\usepackage{caption}
\usepackage{subcaption}
\DeclareGraphicsExtensions{.png, .pdf, .jpeg, .jpg}

\usepackage{amsmath}
\usepackage{bm}
\DeclareMathOperator*{\argmax}{argmax}
\DeclareMathOperator*{\argmin}{argmin}
\DeclareMathOperator{\Tr}{Tr}
\DeclareMathOperator{\Cov}{Cov}
\DeclareMathOperator{\Var}{Var}
\DeclareMathOperator{\E}{E}
\DeclareMathOperator\supp{supp}
\DeclareMathOperator{\sign}{sign}
\DeclareMathOperator{\const}{const.}

\usepackage{amssymb}
\usepackage{wasysym}
\usepackage{mathtools}
\usepackage{gensymb}
\usepackage{tensor}
%\usepackage{langle|}
\usepackage{bbold}

\usepackage{wrapfig}

%eqn environments
\newcommand{\be}{\begin{equation}}
\newcommand{\ee}{\end{equation}}
\newcommand{\eq}[1]{\begin{align*}#1\end{align*}}
\newcommand{\numeq}[1]{\begin{align}#1\end{align}}
\newcommand{\bp}{\begin{pmatrix}}
\newcommand{\ep}{\end{pmatrix}}

%vectors
\newcommand{\rm}[1]{\mathrm{#1}}
\newcommand{\mc}[1]{\mathcal{#1}}
\newcommand{\wv}{\mathbf{w}}
\newcommand{\xv}{\mathbf{x}}
\newcommand{\Xv}{\mathbf{X}}
\newcommand{\yv}{\mathbf{y}}
\newcommand{\zv}{\mathbf{z}}
\newcommand{\Iv}{\mathbf{I}}
\newcommand{\Kv}{\mathbf{K}}
\newcommand{\kv}{\mathbf{k}}
\newcommand{\phiv}{\bm{\phi}}
\newcommand{\Phiv}{\mathbf{\Phi}}
\newcommand{\muv}{\bm{\mu}}
\newcommand{\Sigmav}{\mathbf{\Sigma}}
\newcommand{\norm}[1]{\left\lVert#1\right\rVert}

%other maths
\newcommand{\pd}{\partial}
\newcommand{\inprod}[2]{\langle#1,#2\rangle}
\newcommand{\half}{\frac{1}{2}}
\newcommand{\e}{\mathrm{e}}
\newcommand{\eps}{\epsilon}

%statistics
\newcommand{\mvnprefactor}{\frac{1}{(2\pi)^{d/2}|\Sigmav|^{1/2}}}

%probability
\newcommand\given[1][]{\:#1\vert\:}


\newcommand{\convexpath}[2]{
[   
    create hullnodes/.code={
        \global\edef\namelist{#1}
        \foreach [count=\counter] \nodename in \namelist {
            \global\edef\numberofnodes{\counter}
            \node at (\nodename) [draw=none,name=hullnode\counter] {};
        }
        \node at (hullnode\numberofnodes) [name=hullnode0,draw=none] {};
        \pgfmathtruncatemacro\lastnumber{\numberofnodes+1}
        \node at (hullnode1) [name=hullnode\lastnumber,draw=none] {};
    },
    create hullnodes
]
($(hullnode1)!#2!-90:(hullnode0)$)
\foreach [
    evaluate=\currentnode as \previousnode using \currentnode-1,
    evaluate=\currentnode as \nextnode using \currentnode+1
    ] \currentnode in {1,...,\numberofnodes} {
  let
    \p1 = ($(hullnode\currentnode)!#2!-90:(hullnode\previousnode)$),
    \p2 = ($(hullnode\currentnode)!#2!90:(hullnode\nextnode)$),
    \p3 = ($(\p1) - (hullnode\currentnode)$),
    \n1 = {atan2(\y3,\x3)},
    \p4 = ($(\p2) - (hullnode\currentnode)$),
    \n2 = {atan2(\y4,\x4)},
    \n{delta} = {-Mod(\n1-\n2,360)}
  in 
    {-- (\p1) arc[start angle=\n1, delta angle=\n{delta}, radius=#2] -- (\p2)}
}
-- cycle
}